\documentclass{sig-alternate}
\usepackage{times}
\usepackage{algorithm, algorithmic}
\usepackage{amsmath, amssymb} % , amsthm}
\usepackage{graphicx}
\usepackage{algorithm, algorithmic}
% \usepackage{outlines}
\usepackage{color}

\newenvironment{tighterdescription}%
  {\begin{description}%
    \setlength{\itemsep}{1pt}%
    \setlength{\parskip}{1pt}}%
  {\end{description}}

\newenvironment{tighterenumerate}%
  {\begin{enumerate}%
    \setlength{\itemsep}{1pt}%
    \setlength{\parskip}{1pt}}%
  {\end{enumerate}}

\newenvironment{tighteritemize}%
  {\begin{itemize}%
    \setlength{\itemsep}{1pt}%
    \setlength{\parskip}{1pt}}%
  {\end{itemize}}

\begin{document}

% \newtheorem{theorem}[definition]{Theorem}

\title{Draft of Theta-Sketch Paper}

\numberofauthors{1}
\author{three authors}

\maketitle
\begin{abstract}
no abstract yet
\end{abstract}

\section{Story Strategy}

In this version of the story, we don't need to make a big deal out of the simplicity claim.
Instead we claim that our solution is just plain better
than the obvious baselines in almost every way. What are those baselines?

\begin{itemize}
\item Beyer is the only true competitor. They described a KMV-based sketch mart with all of the needed capabilities.
We beat it in almost every way.  Less memory and CPU when creating base sketches. Lower variance for the base sketch estimator. Lower variance for
the setops estimator.
\item HLL is not in the running at all because the estimates for setops would be so bad.
\item Not sure what to say about Chen Cao Bu.
\item Although no one ever has spelled it out in these terms, it would be possible after reading Gibbons to come up with a sketch-mart scheme that is similar to ours, but with $\alpha$ = 1/2,
and with a different rule for when to multiply $\theta$ by $\alpha$. That hypothetical scheme is better than Beyer, but is not as good as our actual scheme.
\end{itemize}

\section{Big-Data Sketch Mart: Motivation and Requirements}

Lee is working on this section.

\section{Our Solution}

Our sketch-mart solution has several main ingredients.

\begin{enumerate}
\item A Sketch Data Structure $(\theta, S)$.
\item A practical Algorithm X, whose inputs are a size target $k$ and a stream $\mathcal{A}$ of size $n$,
and whose outputs are a sketch $(\theta, S)$ and an estimate $Z$ of $n$. These outputs have the following desirable properties:
$E(|S|) = k$, $\sigma^2(|S|) < k/2 + 1/4$, $E(Z) = n$, and
$\sigma^2(Z) < n^2/2k$. Furthermore, the amortized update time of Algorithm X is $O(1)$ with excellent constant factors.
\item A modified Algorithm Y, whose inputs are a size target $k$ and a stream $\mathcal{A}$ of size $n$,
and whose output is a sketch $(\theta, S')$ which has the theoretically convenient property that the
value of $\theta$ is independent of the hash function which produced the elements of $S'$. 
\item A rule for evaluating an arbitrary set expression over a collection of sketches, whose output is not a fact, but rather a sketch.
This allows a sub-expression of a larger expression to be evaluated as soon as the data becomes available.
\item An estimator which is unbiased (that is, whose expected value is $n$)
for a very wide class of sketches $(\theta, S)$. However, the estimator's variance depends on the algorithm
that was used to create the sketch:
\begin{enumerate}
\item If the sketch was produced by KMV with size target $k$, then the estimator's variance is $\sigma^2 < n^2/(k-2)$.
\item If it was produced by Algorithm X with size target $k$, then the estimator's variance is $\sigma^2 < n^2/(k-(1/2))$.
\item If it was produced by Algorithm Y with size target $k$, then the estimator's variance is $\sigma^2 < n^2/k$
\item If the sketch $(\theta, S)$ was produced by the Set Expression Rule applied to a collection of sketches produced by Algorithm Y, 
then, conditioning on the sketch's particular value of $\theta$, the estimator's variance is $\sigma^2 = (n/\theta) - n$.
\end{enumerate}
\end{enumerate}

\section{New Algorithm}

Because of the larger systems context, we will focus on algorithms which naturally\footnote{Although they obviously can be constructed,
we will not be discussing hybrids which e.g. subdivide the memory budget between an instance of KMV and an instance of HLL.}
provide a co-ordinated sample of approximately $k$ hash values that can be used to estimate the cardinality of a set expression evaluated
over a collection of streams. We will call this family of algorithms the ``KMV family'', even though it includes some algorithms (i.e. Gibbons-style
sampling algorithms) that are traditionally not lumped together with KMV. We note that this ``KMV family'' does not include HLL and its variants.

One of our main results will be a new sketching algorithm whose estimate of $n$ has variance < $n^2/(2k)$. This is the lowest ever for
an algorithm belonging to the ``KMV family''. The algorithm is based on a novel counting process that is different those used in the actual
KMV algorithm, or in KMV variants such as Giroire, or in Gibbons-style sampling algorithms, or in HLL for that matter.

We will call this the ``Basic Algorithm''. PseudoCode is shown in Figure ??. 
Its arguments are a size target ``k'', a hash function, and a stream of identifiers.
Purely to reduce notational clutter in the pseudocode and its analysis, we pretend that the
first step is to apply the hash function to the stream of identifiers, producing a stream of hashes.
The next three lines are a high level description of the algorithm's initialization phase,
which processes just enough of the stream to find the first $k$ unique hash values, which are used
to initialize the ``sample'' set $S$. The remainder of the stream of hashes is then processed by
the remainder of the program.

\begin{algorithm}[t]
\caption{Basic Algorithm for Processing a Stream}
\label{alg:basic-algorithm}
\begin{algorithmic}[1]{\footnotesize
\STATE Inputs: (target size $k$, HashFunction(), StreamOfIdentifiers).
\STATE StreamOfHashValues $ \leftarrow$  HashFunction(StreamOfIdentifiers).
\STATE prefix $\leftarrow$ shortest prefix of StreamOfHashValues containing \\ 
       \quad \quad \quad \quad \quad $k$ unique hash values.
\STATE restOfStreamOfHashValues $\leftarrow$ the corresponding suffix.
\STATE $S \leftarrow$ the $k$ unique hash values in prefix.
\STATE $\alpha \leftarrow k/(k+1)$.
\STATE $i \leftarrow 0$.
\FORALL{$h$ in restOfStreamOfHashValues}
% \STATE \COMMENT{Invariant: $S$ is exactly the set of all $h < \alpha^i$ \\ \quad \quad \quad 
% seen so far in restOfStreamOfHashValues}
\STATE \COMMENT{Invariant: $E(|S|) = k$}.
\IF{$h < \alpha^i$ AND $h \not \in S$}
\STATE $S \leftarrow S \cup \{h\}$.
\STATE $i \leftarrow i + 1$.
\STATE $S \leftarrow S \setminus \{(x \ge \alpha^i) \in S\}$.
\ENDIF
\ENDFOR
\STATE Output the estimate $Z = k / \alpha^i$ and the sketch $(\theta = \alpha^i, S)$.
}\end{algorithmic}
\end{algorithm}



\subsection{Overview Of Analysis}

We will be proving a number of theorems about the basic algorithm, and also about a 
couple of variants (not mentioned yet) of the basic algorithm.

In section ??, we will lay some groundwork in the form of  
a technical lemma about a function $g()$ which encodes the 
behavior of several summations involving the probability distribution
which governs the value of $i$ during the execution of the basic algoritm.

In section ?? we will use this groundwork to analyze the estimate $Z = k / \alpha^i$. We will
prove that it is unbiased, that its variance is ??, the best ever for
an algorithm from the ``KMV familty''. In fact, we will prove that all
of the moments of its distribution are well behaved, resulting in a
strong bound of the epsilon-delta variety. 

In section ?? we will prove that the invariant mentioned in line ?? of
the ``Basic Algorithm'' is indeed true.  That is, $E(|S|) = k$ at the
top of the ``for loop''.  We will also bound the variance of the size
of $S$. These results are needed to show the memory and CPU resources
consumed by this algorithm are roughly similar to (or better than)
other algorithms in the KMV family.

In section ?? we will analyze a different estimator $X=|S|/\alpha^i$ 
which is related to the ``practical'' version of our setops scheme. 
This analysis will re-use the technical lemma about the function ``$g()$'', and will 
also re-use the analysis of the size of the set $S$.

In section ?? we will analyze yet another estimator $Y$ which is related to
the ``theoretical'' version of our setops scheme. This analysis
will re-use the technical lemma about the function ``$g()$''.


\subsection{Groundwork for the Analysis}

In this section we lay the groundwork for the next section's analysis
of the estimator $Z = k / \alpha^i$.

% In this section and the next one, we will simplify the presentation by
% noting that aside for allowing duplicates to be detected and removed,
% the set $S$ plays no role in the computation of the estimator $Z$. We
% would get the same answer by feeding an {\em already} de-duplicated stream
% of hashes into Algorithm 2, which is the same as Algorithm 1 except that
% there is no set $S$ available for doing the (now unnecessary) de-duplication.

% \begin{algorithm}[t]
% \caption{Simplified Code for Analyzing Estimator Z}
% \label{alg:counting-algorithm}
% \begin{algorithmic}[1]{\footnotesize
% \STATE Inputs: (target size $k$, StreamOfUniqueHashValues).
% \STATE Skip the first $k$ items in StreamOfUniqueHashValues.
% \STATE $\alpha \leftarrow k/(k+1)$.
% \STATE $i \leftarrow 0$.
% \FORALL{$h$ in restOfStreamOfUniqueHashValues}
% \IF{$h < \alpha^i$}
% \STATE $i \leftarrow i + 1$.
% \ENDIF
% \ENDFOR
% \STATE Output the estimate $Z = k / \alpha^i$.
% }\end{algorithmic}
% \end{algorithm}

We need to establish some assumptions and some notation. For the
purposes of the analysis, we are assuming that the hash function
produces perfectly random real values on the interval $[0,1)$.  Hence
the probability of a hash collision is zero. The symbol $n$
denotes the number of unique identifiers in the input stream. All of
the algorithms in this section handle the first $k$ unique
identifiers during an initialization phase. Hence the number of unique
identifers processed by the main loop of each algorithm is $u = n - k$.
The symbol $\alpha$ always denotes the fraction $k/(k+1)$.

\subsubsection{The distribution \;Pr(\;i\;|\;u\;)}
% $\mathrm{Pr}(\;i\;|\;u)$}
Now, let us analyze the probability distribution $Pr(i|u)$ which
governs the distribution of the final value of $i$ when one runs
% either the basic algorithm or the simplified algorithm
the basic algorithm
on a stream containing $n = k + u$ unique identifiers. We will characterize this
distribution via a recurrence and base cases.  The base cases are 
easy to see. If $u = 0$, then the algorithm's ``for loop'' is skipped,
and the final value of $i$ is zero. Hence $Pr(i=0|u=0) = 1$, and $Pr(i |u=0) = 0, \forall i > 0$.
On the other hand, if $u > 0$, then the body of the ``for loop'' will be
executed at least once, and the test in line ?? will succeeed at least once
(because a random value from $[0,1)$ must be less than $1 = \alpha^0$).
Therefore $i \geq 1$, so for all $u > 0$, $Pr(i=0 |u) = 0$.

The recurrence can be worked out by considering a state space in which the pair $(i,u)$
means that $i$ increments have occured, and  $u$ unique hash values have been processed.

Suppose that the algorithm ends up in state $(i, u)$. Before processing the final
unique hash value and reaching this state, the algorithm must have been in one of 
two states: either $(i, u-1)$, or $(i-1, u-1)$. We will make the
induction hypothesis that we already know the probabilities of reaching those two states.

Now we need to know the probability of transitioning from 
$(i, u-1)$ to $(i, u)$, and the probability of transitioning from 
$(i-1, u-1)$ to $(i, u)$.

A transition from $(i, u-1)$ to $(i, u)$ occurs when the test in line ??
fails, that is, when $h \geq \alpha^i$.
Because $h$ is a uniform random variable on $[0,1)$, this happens
with probability $(1 - \alpha^i)$. Hence the probability of reaching state $(i,u)$ 
by going through state $(i, u-1)$ is $(1 - \alpha^i)$ times 
the probability of reaching state $(i,u-1)$.

A transition from $(i-1, u-1)$ to $(i, u)$ occurs when the test in line ??
succeeds, that is, when $h < \alpha^{i-1}$.
Because $h$ is a uniform random variable on $[0,1)$, this happens
with probability $\alpha^{i-1}$.
Hence the probability of reaching state $(i,u)$ 
by going through state $(i-1, u-1)$ is $(1 - \alpha^i)$ times 
the probability of reaching state $(i-1,u-1)$.

The total probability of reaching (and hence ending up in) state $(i,u)$ is then 
the sum of the probabilities of reaching it via these two routes. 
This establishes the recurrence, which we now explicitly state
along with the above-mentioned base cases:
\begin{align}
\mathrm{Pr}(0 | 0) = & 1 \\
\mathrm{Pr}(i | 0) = & 0, \quad \forall i > 0 \\
\mathrm{Pr}(0 | u) = & 0, \quad \forall u > 0 \\
\mathrm{Pr}(i | u) = & (1 - \alpha^i) \cdot \mathrm{Pr}(i | u\!-\!1) \; + \\
 & \quad \quad  \alpha^{i\!-\!1} \cdot \mathrm{Pr}(i\!-\!1 | u\!-\!1), \quad \forall i\!>\!0, \forall u\!>\!0
\end{align}

\subsubsection{The Function g(q,k,u)}

The analysis of the Estimator Z involves several summations involving
the probability distribution $Pr(i|u)$. As part of the groundwork for
that analysis, we will prove a technical lemma about a function
$g(q,k,u)$ which encodes the values of those summations.  This section
is completely technical and can be safely skipped by readers who are
more interested in the high-level story than in the details of the proofs.
Here is the definition of $g(q,k,u)$:

\begin{equation}
g(q,k,u) = \sum_{i=0}^u \frac{1}{\alpha^{qi}} \mathrm{Pr}(i|u)
\end{equation}

% \noindent We remind readers that $\alpha = k/(k+1)$.

\noindent {\bf Technical Lemma:} The function $g(q,k,u)$ defined by equation ??
satisfies the following base cases and recurrence:
\begin{align}
g(0,k,u) = & \;1 \\
g(q,k,0) = & \;1 \\
g(q,k,u\!+\!1) = & g(q,k,u) + \left(\frac{1-\alpha^q}{\alpha^q}\right) \cdot g(q\!-\!1,k,u)
\end{align}
\begin{proof}
The base cases can be verified by inspection. The recurrence can be derived from the
recurrence for $Pr(i|u)$ as follows:
\begin{align} 
  g(q,k,u\!+\!1)
= & \sum_{i=0}^{u+1} \frac{1}{\alpha^{qi}} \mathrm{Pr}(i|u+1) \\
= & \left[ \sum_{i=0}^{u} \frac{1}{\alpha^{qi}} (1-\alpha^i) \mathrm{Pr}(i|u) \right] + 0 \\
  & \quad + 0 + \left[ \sum_{i=1}^{u+1} \frac{1}{\alpha^{qi}} (\alpha^{i-1}) \mathrm{Pr}(i\!-\!1|u) \right]
\end{align} 
\noindent Now we will consider the two bracketed sums. For the first one:
\begin{align} 
  & \sum_{i=0}^{u} \frac{1}{\alpha^{qi}} (1-\alpha^i) \mathrm{Pr}(i|u) \\
= & \sum_{i=0}^u \frac{1}{\alpha^{qi}} \mathrm{Pr}(i|u) - \sum_{i=0}^u \frac{\alpha^i}{\alpha^{qi}} \mathrm{Pr}(i|u) \\
= & g(q,k,u) - \sum_{i=0}^u \frac{1}{\alpha^{(q-1)i}} \mathrm{Pr}(i|u) \\
= & g(q,k,u) - g(q-1,k,u).
\end{align}
\noindent For the second one: % performing the change of variables $j = i\!-\!1$:
\begin{align} 
  & \sum_{i=1}^{u+1} \frac{1}{\alpha^{qi}} (\alpha^{i-1}) \mathrm{Pr}(i\!-\!1|u) \\
= & \sum_{j=0}^{u} \frac{1}{\alpha^{q(j+1)}} (\alpha^{j}) \mathrm{Pr}(j|u) \\
= & \frac{1}{\alpha^q}\sum_{j=0}^{u} \frac{1}{\alpha^{(q-1)j}} \mathrm{Pr}(j|u) \\
= & \frac{1}{\alpha^q} g(q-1,k,u).
\end{align}
\noindent Adding ?? and ?? yields the claimed result:
\begin{align} 
  & g(q,k,u) - g(q-1,k,u) + \frac{1}{\alpha^q} g(q-1,k,u) \\
= & g(q,k,u) + \left(\frac{1-\alpha^q}{\alpha^q}\right) \cdot g(q\!-\!1,k,u)
\end{align}
\end{proof}

\noindent {\bf Corollaries of Technical Lemma}: For all integers $u \geq 0$: 
\begin{align}
% g(0,k,u) = & \;1 \\
g(1,k,u) = & \;\frac{k+u}{k} \\
g(2,k,u) = & \;\frac{k^3 + 2 k^2 u + ku^2 + u(u\!-\!1)/2}{k^3}
\end{align}
\begin{proof}
$\frac{k+u}{k}$ satisfies the same base case and recurrence as $g(1,k,u)$.
The base case (eqn ??) can be verified by inspection. The recurrence (eqn ??) can be verified as follows:
\begin{equation}
\frac{k+(u+1)}{k} = \frac{k+u}{k} + \frac{1-\alpha}{\alpha} \cdot 1 = \frac{k+u}{k} + \frac{1}{k}.
\end{equation}
$\frac{k^3 + 2 k^2 u + ku^2 + u(u\!-\!1)/2}{k^3}$ satisfies the same base case and recurrence as $g(2,k,u)$.
The base case (eqn ??) can be verified by inspection. The recurrence (eqn ??) can be verified as follows:
\begin{align}
   & g(2,k,u) + \left(\frac{1-\alpha^2}{\alpha^2}\right) \cdot g(1,k,u) \\
 = & g(2,k,u) + \frac{1}{\alpha^2} \cdot \frac{k+u}{k} - \frac{k+u}{k} \\
 = & g(2,k,u) + \frac{(k+1)^2}{k^2} \cdot \frac{k+u}{k} - \frac{k+u}{k} \\
 = & g(2,k,u) + \frac{2k^2+k+2uk+u}{k^3} \\
 = & \frac{k^3 + 2k^2u +2k^2 + ku^2 + 2uk + k + \frac{1}{2} u^2 + \frac{1}{2} u}{k^3} \\
 = & \frac{k^3 + 2k^2(u+1) + k(u+1)^2 + (u+1)u/2}{k^3} \\
 = & g(2,k,u+1).
\end{align}
\end{proof}



\subsection{Analysis of the Estimator Z}

\subsubsection{Mean and Variance}

The expected value and variance of the estimator $Z = k / \alpha^i$ can be computed
with the help of the above technical lemma:

\begin{align}
 E(Z|k,u)  = & \sum_{i=0}^{u}\frac{k}{\alpha^i} \cdot \mathrm{Pr}(i|u) \\
  = & k \cdot g(1,k,u) = k\cdot\frac{k+u}{k} = n \\
\vspace{0.75em}
 E(Z^2|k,u) = & \sum_{i=0}^{u}\frac{k^2}{\alpha^{2i}} \cdot \mathrm{Pr}(i|u) \\
  = & k^2 \cdot g(2,k,u) \\
  = & \frac{k^3 + 2k^2u + ku^2 + u(u-1)/2}{k} \\
  = & \frac{u(u-1)/2}{k} + n^2 \\
\vspace{0.75em}
 \sigma^2(Z|k,u) = & E(Z^2|k,u) - n^2 \\
                 = & \frac{u(u-1)}{2k} \\
                 = & \frac{n^2-2nk+k^2-n+k}{2k} \\
                 < & \frac{n^2}{2k}\quad \mathrm{for\;large} \; \frac{n}{k}.
\end{align}

\subsubsection{Discussion}

The variance of Estimator Z is smaller by a factor of 2 than that of other ``KMV class'' 
algorithms such as KMV itself, Giroire's generalizations of KMV,
various algorithms that sample using a decreasing threshold such as Gibbons and
our our own estimator X, and even sampling with the ``ideal'' fixed
fixed threshold $\theta = k / n$. But why, exactly? We do not know yet.

\subsubsection{Higher Moments and Confidence Intervals}

Anirban will produce this material.

\subsection{Analysis of the Size of S}
% so we need to return from the ``Simplified'' Algorithm 2 to the ``Basic'' Algorithm 1. 
% Now we are going to focus on size of the set $S$

We will now show that the expected value of $|S|$ is $k$, and its variance
is upper bounded by $k/2 + 1/4$. These results are needed to establish that our
basic algorithm is practical, and also to show that estimator $Z$ is achieving
lower variance than other KMV-like algorithms while collecting roughly the same
number of hash values in its sample.

\subsubsection{Intuitive Discussion}

Before getting to the actual analysis, we mention that the variance of $|S|$
is smaller than one might expect from a couple of plausible-sounding but 
inaccurate analogies for what is going on.

The first inaccurate analogy is that the mechanism governing the size of 
$S$ resembles sampling with a fixed threshold $\theta = k/n$. However,
for $n \gg k$, that analogy suggests that $|S|$ has a Poisson distribution; 
the resulting variance of $k$ is too big by a factor of 2.

The second inaccurate analogy assumes that the mechanism is essentially an ordinary random walk in which each step is caused by the
combined effect of lines ?? and ??. However, that would cause the variance of $|S|$ to keep growing forever as $n$ increases;
that answer is wrong by an unbounded factor.

A third analogy is only a little bit inaccurate, namely that the size
of $S$ is governed by a certain AR(1) process, namely a random walk
whose steps are determined by a linear restoring force plus noise.
The linear restoring force arises from the fact that line ?? always
inserts exactly one item, but the expected number of items deleted by
line ?? is less than 1 if $|S| < k$, and more than 1 if $|S| > k$.
When an approximate description of our process consisting of the
actual restoring force plus an approximation of the noise distribution
is plugged into a basic theorem about $AR(1)$ processes, an answer
close to $k/2$ pops out for the ``steady state'' variance of the size
of $S$. This is close to the true answer.

\subsubsection{The Actual Analysis}

In this section we will put aside the above intuitive arguments and
do a proper analysis of the distribution governing $|S|$. This can be done
by analyzing two survival processes, a process {\bf K} governing the
$k$ items which were inserted into $S$ by line ?? of Algorithm ??, and
a process {\bf I} governing the $i$ items which were inserted into $S$
by line ?? of the algorithm.

\vspace{0.5em}
\noindent {\bf Process K:} A hash value $h$ inserted into $S$ by line ?? ends up in the final set $S$ if and only it survives the
$i$ rounds of deletion which subsequently occur in line ??. Since the test in line ?? becomes monotonically more stringent
in successive deletion rounds, this will happen if and only $h$ can survive the final round of deletion, where the test is
$h < \alpha^i$. Because $h$ is a uniform random variable on $[0,1)$, this occurs with probability $\alpha^i$.
Furthermore, because the $h$'s are iid uniform random variables, the indicator variables for their
respective survivals are iid Bernoulli random variables, and the total
number of survivors amongst the $k$ initial members of $S$ is governed by the Binomial Distribution with $k$ draws and
probability $\alpha^i$. Hence the expected number of survivors of process {\bf K} is $k \cdot \alpha^i$, and the variance is $k \cdot \alpha^i \cdot (1 - \alpha^i)$.

\vspace{0.5em}
\noindent {\bf Process I:} This process is not governed by a Binomial
Distribution, because the survival probabilities of the $i$ hashes
inserted by line ?? are all different. 
In all cases, ultimate survival is determined
by same final test $h < \alpha^i$ (in line ??); the explanation
for the differing survival probabilities is that each hash value
passed a different test (in line ??) upon entering the set $S$.
For example, consider the final hash value to enter
the set $S$. Because it passed the entry test $h < \alpha^{i-1}$ in
line ??, it passes the final deletion test $h < \alpha^i$ with
probability $\alpha^i/\alpha^{i-1} = \alpha^1$. Similarly, the
second-to-last hash value to enter the set $S$ passes the final
test with probability $\alpha^i/\alpha^{i-2} = \alpha^2$.  Overall, the survival
of the $i$ hash values can be modelled by $i$ iid Bernoulli
random variables whose respective probabilities are $\alpha^1,
\alpha^2, \ldots, \alpha^i$. The expected number of survivors of
process {\bf I} is $\sum_{j=1}^{i} \alpha^j$, and the variance is
$\sum_{j=1}^{i} \alpha^j (1 -\alpha^j)$.

Now, because all of the survival events are independent, the probability distribution governing the
size of $S$ is the sum of the probability distributions governing process {\bf K} and process {\bf I}.
Therefore:
\begin{align}
E(|S|) = & k \cdot \alpha^i + \sum_{j=1}^{i} \alpha^j 
       = k \cdot \alpha^i + \frac{\alpha - \alpha^{i+1}}{1 - \alpha} \\
       = & k.     
\end{align}
\begin{align}
\sigma^2(|S|) = & k \cdot \alpha^i \cdot (1 - \alpha^i) + \sum_{j=1}^{i} \alpha^j (1 -\alpha^j) \\
              = & k \;\alpha^i \;(1 - \alpha^i) \\
                & \quad + \frac{\alpha + \alpha^{2(i+1)} - \alpha^{i+2} - \alpha^{i+1}}{1 - \alpha^2} \\
             = & \frac{\alpha - \alpha^{2i+1}}{1 - \alpha^2} \\
             = & \frac{\alpha}{1-\alpha^2}(1-\alpha^{2i}) \\
             < & \frac{\alpha}{1-\alpha^2} \\
             = & \frac{k^2+k}{2k+1} \\
             < & \frac{k}{2} + \frac{1}{4}.
\end{align}
% k \cdot \alpha^i
% k \cdot \alpha^i \cdot (1 - \alpha^i)
% \sum_{j=1}^{i} \alpha^j
% \sum_{j=1}^{i} \alpha^j (1 -\alpha^j)

\subsection{Analysis of the Estimator X}

Our basic algorithm outputs not only the stand-alone estimator $Z = k
/ \alpha^i$, but also the theta sketch $(\theta = \alpha^i, S)$ which
is stored in the sketch mart for later use in set expressions. Very
roughly speaking, the accuracy of a set expression estimate is
determined by combination of the estimator $X = |S| / \alpha^i$ and a
penalty factor which grows worse as the true answer grows smaller
relative to the size of the streams. We will discuss the penalty
factor later; right now we will analyze the estimator $X$. We will
prove that $X$ is unbiased, and that the variance of $X$ is slightly
less than the variance of KMV.

In the following, $B$ is the random variable for the size of $|S|$.

\begin{align}
 & E(X | k, u)  \\
= & \sum_{i=0}^u \sum_{b=0}^{k+i}
\frac{b}{\alpha^i} \cdot \mathrm{Pr}(B \!=\! b | k, i) \cdot
\mathrm{Pr}(I \!=\! i | u) \\
= &\sum_{i=0}^u \frac{1}{\alpha^i} \mathrm{Pr}(I \!=\! i | u) \cdot 
\sum_{b=0}^{k+i} b \cdot \mathrm{Pr}(B \!=\! b | k, i) \\
= & \sum_{i=0}^u \frac{1}{\alpha^i} \mathrm{Pr}(I \!=\! i | u) \cdot k \\
= & k \cdot g(1,k,u) = k \cdot \frac{k+u}{k} = k + u \\
= & n.
\end{align}

% \vspace{1.5em}

\begin{align}
 & E(X^2 | k, u) \\
= & \sum_{i=0}^u \sum_{b=0}^{k+i} \left( \frac{b}{\alpha^i} \right)^2 \mathrm{Pr}(b|k,i) \; \mathrm{Pr}(i|u) \\
= & \sum_{i=0}^u \frac{1}{\alpha^{2i}} 
\mathrm{Pr}(i|u)
\sum_{b=0}^{k+i} b^2 
\mathrm{Pr}(b|k,i) \\
= & \sum_{i=0}^u \frac{1}{\alpha^{2i}} 
\mathrm{Pr}(i|u)
\left(  \frac{\alpha-\alpha^{2i+1}}{1-\alpha^2} + k^2  \right) \\
= & \frac{1}{1-\alpha^2} 
\left[
\alpha \sum_{i=0}^u \frac{1}{\alpha^{2i}} \mathrm{Pr}(i|u) -
\sum_{i=0}^u \alpha \mathrm{Pr}(i|u)
\right] \\
 & \quad \quad + k^2 \sum_{i=0}^u \frac{1}{\alpha^{2i}} \mathrm{Pr}(i|u) \\
= & \frac{1}{1-\alpha^2} 
\left[ \alpha \cdot g(2,k,u) - \alpha \cdot 1 \right] + k^2 g(2,k,u) \\
= & (\frac{\alpha}{1-\alpha^2} + k^2) \cdot g(2,k,u) - \frac{\alpha}{1-\alpha^2} \\
= & \frac{k(2k^2+2k+1)}{2k+1}
\cdot g(2,k,u) - 
\frac{k(k+1)}{2k+1} \\
= & \frac{
\left(\begin{array}{c}
2k^5 + 4k^4u + 2k^3u^2 + 3k^2u^2 + k^4 + 4k^3u \\
\quad \quad \quad + 2ku^2 + k^2u - ku + u(u-1)/2
\end{array}\right)
}{k^2 (2k+1)} \\
= & \frac{k^2u + ku^2 + u(u\!-\!1)/2 + k^4 + 2k^3u + k^2u^2}{k^2}.
\end{align}

% \vspace{1.5em}

\begin{align}
 \sigma^2 (X|k,u) 
 = & E(X^2|k,u) - E^2(X|k,u) \\
 = & \frac{k^2u + ku^2 + u(u\!-\!1)/2}{k^2} \\
 = & \frac{(2k+1)n^2 - (2k^2+2k+1)n + (k^2+k)}{2k^2} \\
       \rightarrow & \frac{(2k+1)n^2}{2k^2} \\
 = & \frac{n^2}{k-\frac{k}{2k+1}} \\
 < & \frac{n^2}{k-\frac{1}{2}}.
\end{align}


\subsection{Analysis of the Estimator Y}

Although our actual sketch mart system computes estimates for set expressions
from theta sketches produced by the Basic Algorithm, and this works
fine in practice (see the experiments section below), it appears
that a formal analysis of the resulting errors would be very complicated.

So, strictly for the purposes of enabling a simple error analysis for
set expressions, we now present a modified version of the basic
algorithm, which produces a different theta sketch $(\theta_1 = \alpha^i,
S_2)$, and a corresponding estimator $Y = |S_2| / \theta_1$. We will
prove that $Y$ is unbiased, and that the variance of $Y$ is slightly
less than the variance of KMV.

The modification consists of a second pass over the stream, using a
second hash function that is independent of the first hash function.
During this second pass, we collect into the set $S_2$ all (second) hash
values that are less than the threshold $\theta_1 = \alpha^i$ that was computed
using the first hash function.\footnote{It is not hard to figure out how to obtain the same
output using a {\em single} pass using both hash functions. However, the
resulting algorithm would still be twice as expensive as the
unmodified basic algorithm in terms of both memory and CPU; that is
why we don't actually do it in practice.} This extra work ensures that
the theta-sketch's threshold $\theta_1$ and sample set $S_2$ are independent.
This causes the size of $S_2$ to be governed by the Binomial distribution
with $n$ throws and probability $\theta_1$, which greatly simplifies the
error analysis for set expressions. First, though, we will analyse the single-sketch
estimator $Y = |S_2| / \theta_1$.

In the following, $C$ is the random variable for the size of $|S_2|$.

\begin{align}
 & E(Y|k,u)  \\
= &    \sum_{i=0}^u    \sum_{c=0}^{n}    \frac{c}{\alpha^i}   \cdot \mathrm{Pr}(C = c|n,i)   \cdot \mathrm{Pr}(i|u) \\
= &    \sum_{i=0}^u  \frac{1}{\alpha^i}  \mathrm{Pr}(i|u)  \cdot \sum_{c=0}^{n} c \; \mathrm{BinomialPMF}(c|n,\alpha^i) \\     % \mathrm{Pr}(c|n,i)  
= &    \sum_{i=0}^u  \frac{1}{\alpha^i}  \mathrm{Pr}(i|u)  \cdot n \; \alpha^i \\
= & n \cdot \sum_{i=0}^u \mathrm{Pr}(i|u) = n.
\end{align}

\begin{align}
  & E(Y^2|k,u)  \\
= &    \sum_{i=0}^u    \sum_{c=0}^{n}    \frac{c^2}{\alpha^{2i}}   \cdot \mathrm{Pr}(C = c|n,i)   \cdot \mathrm{Pr}(i|u) \\
= &    \sum_{i=0}^u  \frac{1}{\alpha^{2i}}  \mathrm{Pr}(i|u)  \cdot \sum_{c=0}^{n} c^2 \; \mathrm{BinomialPMF}(c|n,\alpha^i) \\     % \mathrm{Pr}(c|n,i)  
= &    \sum_{i=0}^u  \frac{1}{\alpha^{2i}}  \mathrm{Pr}(i|u)  \left[ n \alpha^i - n \alpha^{2i} + n^2 \alpha^{2i} \right] \\
= &  n \sum_{i=0}^u  \frac{1}{\alpha^{i}} \mathrm{Pr}(i|u) - n \sum_{i=0}^u \mathrm{Pr}(i|u) + n^2 \sum_{i=0}^u \mathrm{Pr}(i|u) \\
= & n \cdot g(1,k,u) - n + n^2 \\
= & \frac{n^2}{k} - n + n^2.
\end{align}

\begin{align}
 \sigma^2(Y|k,u) = E(Y^2|k,u) - n^2 = \frac{n^2}{k} - n.
\end{align}

Interestingly, this is the same variance as the fixed threshold algorithm run with $\theta = \frac{k}{n}$.
However, the two distributions are {\em not} actually the same, despite having the same mean and variance.

% \noindent line ?? is because $E(c^2) = \sigma^2(c) + E^2(c) = np(1-p) + (np)^2$.

\section{Implementation of Basic Algo}

The writeup in this section is somewhat out of date.

\subsection{Cost of Executing the Algorithm}

We will now analyze the execution cost of Algorithm 1, assuming that the
set S is implemented using a hash table of size $c_h \cdot k$, and
assuming that this hash table supports $O(1)$-cost lookup and insert operations.
% \footnote{The actual constant factor in this $O(1)$ depends on the
% specific type of hash table, and also on the headroom factor $c_h$.}
% \footnote{The hash table doesn't need to support $O(1)$
% deletions of specific items; we can't get our hands on those items
% anyway without the priority queue.} 

Suppose that the stream contains $D$ duplicate hashes in addition
to the $n$ unique hashes that we have been assuming so far. The cost of processing
these is $O(D)$; this cost is incurred in lines 1-9.

Now, let us consider the $n=w+u$ unique
hashes in the stream.  These incur a cost of $O(n)$ in lines 1-9. 
In addition, $t$ of the $u$ unique hashes processed by the main loop
make it past line 9 and cause the sketch to be updated in lines 10-12. We will not
formally analyze the number of updates $t$, focusing instead on the cost of
performing each update.\footnote{Informally, $E(t)$ is roughly 
$k \ln (\frac{u+k}{k})$ for large $k$ (assuming that $\alpha = \frac{k}{k+1}$).
This is similar to the number of updates performed by the heap-based KMV algorithm.}
% [cite Beyer] could have tightened their bound on the cost of the heap-based algorithm
% by including the division by $k$ inside the logarithm.}

Clearly, each update incurs an $O(1)$ cost in lines 10 and 11. 
However, line 12 is a problem. We want to delete from $S$ those
hashes that are not less than the newly reduced $\theta$. Even if the
hash table had an $O(1)$-cost deletion operation, we could not easily 
use it; without the priority queue we do not know of an inexpensive way 
to immediately locate the specific items that need to be deleted.

Our solution to this problem is to leave the ``deleted'' items in the hash table for a while.
They are eventually removed during occasional rebuilds of the hash table. Each rebuild costs $O(k)$,
so we need to make sure that the table isn't rebuilt more often than once per $O(k)$ inserts.
Having ensured that, the lazy version of line 12 has an amortized cost of $O(1)$, and we are done.

\subsubsection{Further Discussion of the Lazy Algorithm}

Pseudocode for the lazy version of our algorithm is shown in figure 2. Instead of maintaining the 
set S exactly, this version maintains a set $T \supseteq S$ which not only contains $S$ but also
some other hashes that are bigger than $\theta$ but haven't been deleted yet by the next rebuild of the 
hash table which implements $T$.

Because the algorithm is tracking the larger set $T$ but needs to return the size of $S$ in line 6, 
some extra work is needed there; one must count the number of elements of $T$ which are less than the current
value of $\theta$.

Now we will discuss the set size $c_r \cdot k$ which triggers a rebuild of the hash table implementing $T$.
Recall that $c_b \cdot k$ is a high probability upper bound on the maximum size that $S$ an attain during the 
execution of the algorithm, and that $c_h \cdot k$ is the number of slots in the actual hash table. The value
of $c_r$ needs to satisfy $c_b < c_r < c_h$, ideally with a large gap at each inequality.

The gap in the inequality $c_r < c_h$ ensures that the hash table's occupancy never becomes too large,
thus improving the constant factor in the assumed $O(1)$ cost of hash table operations.

The gap in the inequality $c_b < c_r$ ensures that table rebuilds do not occur too often, thus improving the constant
factor in the $O(1)$ amortized cost of line 12.

\begin{algorithm}[t]
\caption{Basic Algorithm with Lazy Deletion$(\mathcal{H},k,c_r)$}
\label{alg:basic-algo-lazy}
\begin{algorithmic}[1]{\footnotesize
\STATE $\alpha = k/(k+1)$.
\STATE $T \leftarrow\;$ (First $k$ unique hashes in $\mathcal{H}$).
\STATE $ \theta \leftarrow 1.0 $
\STATE {\bf Loop:}
% \STATE \COMMENT{Invariant: $E(|T|) >= \alpha/(1-\alpha)$}.
% \STATE \COMMENT{Invariant: $T$ contains at least all $h < \theta$ seen so far in $\mathcal{H}$}.
\STATE \COMMENT{Invariant: $T \supseteq S$}.
% \STATE \COMMENT{Invariant: $|S| \le |T| < c_r \cdot k$}.
\STATE {\bf if} no more hashes in $\mathcal{H}$, {\bf return} $(\theta,|\{(h < \theta) \in T\}|)$.
% \STATE Answer pending queries using $(\theta, |\{(h < \theta) \in T\}|)$.
\STATE $h \leftarrow\;$ next hash in $\mathcal{H}$.
\STATE {\bf if} $h \ge \theta$ {\bf goto Loop}.
\STATE {\bf if} $h \in T$ {\bf goto Loop}.
\STATE $T \leftarrow T \cup \{h\}$.
\STATE $\theta \leftarrow \alpha \cdot \theta$.
\STATE {\bf if} $|T| >= c_r \cdot k$ {\bf then} $T \leftarrow \{(h < \theta) \in T\}$.
\STATE {\bf goto Loop.}
}\end{algorithmic}
\end{algorithm}

\section{Implementation}

Now that we have Algorithm 2, the main thing to decide is the specific kind of hash table
to use in the actual implementation. The following factors influenced our decision.

\begin{tighterenumerate}

\item Because the items to be stored are themselves hash values, a simple array of hash values accessed via open addressing is
more space efficient than a library hash table that stores items, keys, and perhaps hashed keys.
\item Because we employ high quality hash functions (City Hash and Murmur Hash)
that have good randomness in not only the high but also the low
bits of the hash values, we can safely use open addressing with a power-of-two hash table size, thus 
allowing us to probe the table using masked-off low bits of the hash values.
\item Because our algorithm is already handling deletions lazily via occasional table rebuilds,
the hash table does not need to support the efficient deletion of specific items. Hence we are
free to use a double hashing version of open addressing which gracefully handles high load factors
that permit the rebuilds to be done less often.
\end{tighterenumerate}

\subsection{Optimizations}

There is some redundant work in Algorithm 2 which can be avoided by
using a single sequence of probes to achieve the combined effect of
the lookup operation in line 9, and the possible insert in line 10.

As a further optimization, one can cause rebuilds to happen less often
by modifying the implementation of the insert operation so that
whenever possible it overwrites an ``expired'' item (that is, an $h$ that is
not less than the current threshold) instead of filling up a previously empty
slot. 

Pseudocode which reflects both of these optimizations appears as
Algorithm 3. There are some tricky details involved in the implementation
of the procedure OptimizedLookup().\footnote{Similar issues are thoroughly discussed in
d in Section ?? of ??.}
In particular, because we are using double hashing, and because the overwriting insert
operation treats expired hashes as if they were already deleted, the necessary
sequence of probes for the lookup and for the insert are not quite the same. This is
because the expired hashes are stepping stones
which preserve the connectivity of the chains of probes leading to other hashes in the table.

Let $h$ be the second argument to OptimizedLookup (). Then the lookup operation's sequence of probes cannot stop until
it finds $h$ or reaches a slot that is actually empty. However, the insert operation's sequence of probes
can stop upon reaching either an empty slot or a slot containing an expired hash.

Our method of obeying both of these rules with a single sequence of probes is to keep going
until reaching $h$ or an empty slot. However, during this sequence of probes, if the procedure
notices any slot containing an expired hash, it remembers the first such slot. At the end
of the sequence of probes, OptimizedLookup() returns (true,null) if it found $h$, or 
(false,locationOfFirstExpiredHash) if it noticed any expired hashes, else 
(false,locationOfEmptySlot).

One final detail is that we are calling OptimizedLookup() with the
third argument $\alpha \cdot \theta$, so it considers any $h \ge
\alpha \cdot \theta$ to be expired. This is anticipating the reduction in 
$\theta$ that will occur in line 13 on the code path which is executed in the
event that $h$ is not found in the table, and is therefore inserted, and therefore needs
a slot to be inserted into.
This code is correct even in the strange-seeming case where an $h$ satisfying
$\alpha \cdot \theta \le h < \theta$ is inserted into the table, possibly overwriting another 
$\alpha \cdot \theta \le h'$.


\begin{algorithm}[t]
\caption{Optimized Algorithm With Lazy Deletion $(\mathcal{H},k,c_r)$}
\label{alg:optimized-algo}
\begin{algorithmic}[1]{\footnotesize
\STATE $\alpha = k/(k+1)$.
\STATE $T \leftarrow\;$ (First $k$ unique hashes in $\mathcal{H}$).
\STATE $ \theta \leftarrow 1.0 $
\STATE {\bf Loop:}
\STATE \COMMENT{Invariant: $T \supseteq S$}.
\STATE {\bf if} no more hashes in $\mathcal{H}$, {\bf return} $(\theta,|\{(h < \theta) \in T\}|)$.
\STATE $h \leftarrow\;$ next hash in $\mathcal{H}$.
\STATE {\bf if} $h \ge \theta$ {\bf goto Loop}.
% \STATE $(foundIt,InsertHere) \leftarrow \mathrm{fancyCheckTable}(T,h,\alpha \cdot \theta)$.
\STATE $(foundIt,InsertHere) \leftarrow \mathrm{OptimizedLookup}(T,h,\alpha \cdot \theta)$.
\STATE {\bf if} $foundit$ {\bf goto Loop}.
% \STATE $T \leftarrow T \cup \{h\}$.
\STATE {\bf if} $\mathrm{Empty}(TableSlot[InsertHere])\;\mathbf{then}\; |T| \leftarrow |T|\!+\!1$.
\STATE $TableSlot[InsertHere] \leftarrow h$.
\STATE $\theta \leftarrow \alpha \cdot \theta$.
\STATE {\bf if} $|T| >= c_r \cdot k$ {\bf then} $T \leftarrow \{(h < \theta) \in T\}$.
\STATE {\bf goto Loop.}
}\end{algorithmic}
\end{algorithm}



\section{Analysis of $Y$ estimator}

\documentclass{sig-alternate}
\usepackage{times}
\usepackage{algorithm, algorithmic}
\usepackage{amsmath, amssymb} % , amsthm}
\usepackage{graphicx}
\usepackage{algorithm, algorithmic}
% \usepackage{outlines}
\usepackage{color}
\newtheorem{theorem}{Theorem}[section]
\newtheorem{lemma}[theorem]{Lemma}
\newtheorem{proposition}[theorem]{Proposition}
\newtheorem{corollary}[theorem]{Corollary}
\newtheorem{definition}[theorem]{Definition}

\newenvironment{tighterdescription}%
  {\begin{description}%
    \setlength{\itemsep}{1pt}%
    \setlength{\parskip}{1pt}}%
  {\end{description}}

\newenvironment{tighterenumerate}%
  {\begin{enumerate}%
    \setlength{\itemsep}{1pt}%
    \setlength{\parskip}{1pt}}%
  {\end{enumerate}}

\newenvironment{tighteritemize}%
  {\begin{itemize}%
    \setlength{\itemsep}{1pt}%
    \setlength{\parskip}{1pt}}%
  {\end{itemize}}

\begin{document}

% \newtheorem{theorem}[definition]{Theorem}

\title{Draft of Theta-Sketch Paper}

\numberofauthors{1}
\author{three authors}

\maketitle
\begin{abstract}
no abstract yet
\end{abstract}
\def\S{{\cal S}}

\section{Single Sketch}
\begin{definition}
For a stream $\S$ define the ``de-duped'' stream $D(\S)$ as follows: if $\S = s_1 s_2\ldots s_n$, then for each $i$, remove all occurrences of $s_i$ from the indices $[i+1, n]$. The resulting stream, where every element appears exactly once, and in the same order as their first appearance in $\S$ is defined to be $D(\S)$.
\end{definition}

The following Lemma shows that our distinct count estimator $Z$ for a single stream has the same guarantees as the Morris counter.
\begin{lemma}
	Let $Z = Z(\S,h)$ be the random variable corresponding to the level when Algorithm 1 is run over stream $\S$ with hash function $h$. Let $M = M(D(\S), \frac{1}{\alpha}, h)$ be the value of the Morris counter when run over $D(\S)$ with base $\frac{1}{\alpha}$ and hash function $h$. Then for any stream $\S$, the random variables $M$ and $Z$ have identical distribution over the random choice of the hash function.  
\end{lemma}

[[Add in proof-- easy but need to be careful about corner cases. ]]

Using the above lemma, and Theorem XXX from Flajolet~\cite{}, we have the following corollary about the expectation and variance of the estimator $Z$. 

\begin{corollary}
For a stream $\S$, the estimate $Z$ has expectation $E[Z] = u$ and variance $\sigma^2(Z)$ bounded by $\sigma^2(Z) < \frac{n^2}{2k}$. Hence, by choosing $\alpha = \frac{k}{k+1}$ where $k = \frac{1}{2\epsilon^2 \delta}$, we have that with probability $ 1 - \delta$, $$ (1 - \epsilon)F_0(S) \le Z \le (1 + \epsilon)F_0(S).$$
\end{corollary}

[[ We need to add in the bound for the space usage-- that is novel.]]


\section{New analysis for Set operations}
\def\bad{\mathsf{bad}}
Suppose we have $m$ streams $A_i$ and $B$ and a set expression $f(\{A_i\})$.
We give bounds on the estimator $Y$ for evaluating the expression. Let $\theta_i$ 
be the threshold values obtained for the $i^{th}$ streams using only the first pass and using two independent private hash functions $h_i$. $\theta_i$, $i\in [1,m]$ are random variables that are independent of each other. We then set $\theta_{r} = \min \theta_i$.


\begin{lemma}
	Let $n_{\max} = \max_i |A_i|$ and let $n_f = |f(\{A_i\})| $
	The estimate $Y$ is unbiased, i.e. $E[Y] = n_f $. 
	If $\alpha = \frac{k}{k+1}$ where $k \ge \frac{1}{2\epsilon^2 \delta}$, then with probability $ 1 - (m+1)\delta$, we have
	$$ n_f - \Delta \le Y \le n_f + \Delta, $$ where $\Delta = \max(\sqrt{\frac{2(1 + \epsilon)n_{\max}}{k}n_f\log(\frac{1}{\delta})}, 1.5(1 + \epsilon)\frac{n_{\max}}{k}\log(\frac{1}{\delta})).$
\end{lemma}

\begin{proof}
	For the given choice of $\alpha$, for a specific $i$, with probability $1 - \delta$, $$(1 - \epsilon)n_i \le \frac{k}{\theta_i} \le (1 + \epsilon)n_i, $$ and hence, by taking union bound over all $i$, with probability $1 - m\delta$.
	$$\frac{k}{(1 + \epsilon)n_i} \le \theta_i \le  \frac{k}{(1 - \epsilon)n_i}. $$
	
Thus, with probability $1 - m\delta$, $\theta = \min_i \theta_i \ge \frac{k}{(1 + \epsilon)n_{\max}}$. We now condition on this event happening. 

For each element $x$ of $ f({A_i})$, let $y_x$ be the indicator variable that indicates whether or not $h(x) < \theta$, i.e. whether or not $x$ is present in the composed sketch. So
\begin{align*}
	y_x = \begin{cases} 1 &\mbox{w.p. } \theta \\
	0 & \mbox{else.}  \end{cases}
\end{align*}
Let $ y = \sum_{i \in f(\{A_i\})} y_i$. Hence $ Y = \frac{y}{\theta}$. 
We then bound $y$ using Chernoff bound. 
$$ \Pr[ |y - n_f \theta| > t ] < \exp\left( - \frac{t^2/2}{n_f\theta(1-\theta) + t/3} \right). $$
By choosing $t = \max(\sqrt{n_f\theta(1 - \theta)\log(\frac{1}{\delta})}, 1.5\log(\frac{1}{\delta}))$, we have that with probability $1 - \delta$,
\begin{align*}
	 | Y - n_f| & < \frac{1}{\theta} \max(\sqrt{n_f\theta(1 - \theta)\log(\frac{1}{\delta})}, 1.5\log(\frac{1}{\delta}))\\
	& < \max(\sqrt{\frac{2(1 + \epsilon)n_{\max}}{k}n_f\log(\frac{1}{\delta})}, 1.5(1 + \epsilon)\frac{n_{\max}}{k}\log(\frac{1}{\delta}))
	\end{align*}
By taking the union bound, the total probability of failure is bounded by $1 - (m+1)\delta$. Hence we have the statement of the Lemma.

\end{proof}


\section{Scratch Space}
We then bound the moment generating function $E[\exp(tM)] = \prod_{i\in A\Delta B} E[\exp(t M_i)]$. Also, $$E[\exp(t M_i) ] = \sum_{j} E[\exp(t M_i) | \theta_{ab} = \alpha_j]P(j | u_a, u_b)$$ where $P(j | u_a, u_b)$ denotes the probability that $\theta_{ab} = \alpha^j$. Also,
\begin{align*}
	E[\exp(tM_i)\; |\; \alpha^j ] = 1 + \alpha^j(e^t - 1).
\end{align*}

The following Lemma follows directly from adapting the Proposition 2 in ~\cite{} to the case of arbitrary bases. 
\begin{lemma}
	For all $i$ and $u$,
	$$p_{iu} < \exp(-k)i (1-\alpha^i)^u.$$
	Hence, for $ i < \ln(n) - \ln\ln n$, 
\end{lemma}

Similarly, the following Lemma follows from Proposition 4 in ~\cite{}.
\begin{lemma}
	For $i = 2\log(n) + \delta$, with $\delta\ge 0$, we have $p_{iu} = O(2^{-\delta}n^{-0.99})$. 
\end{lemma}

Finally, we show the claim. Let $\bad$ denote the event that the estimate is $\epsilon|A\delta B|$ away from the the truth. WLOG, let $u_a \ge u_b$. $J_1 = [1, \log(u_a) - \log\log(u_a)]$,
\begin{align*}
	\Pr[\bad] & \le \sum_{\theta} 
\end{align*}
 
\end{document}


\end{document}








We will now figure what are the transition probabilities

There are only two possible
states which could have preceded that one during the execution of the algorithm, namely




is also fairly easy to see. Let the pair $(i,u)$ denote the state
in which 
Suppose the algorithm is in the state $(i,u)$, meaning that $i$ increments
have occured, and $u$

we end up in the state $(i,u)$, meaning that $i$ increments
have occurred, and $u$ un

There are only two ways
that the algorithm can end up with a given value of $i$ after the main
loop has processed $u$ unique hash values.  The difference lies in
what happened when the $u$'th hash value (numbering from 1 to $u$) was
processed. More specifically, there are two cases: $i$ was not
incremented, or $i$ was incremented.

Consider the case where $i$ was not incremented. Then we know that the
test in line 6 failed, so $h \geq \alpha^i$. Since $h$ is uniformly distributed
on $[0,1)$, this happens with probability $(1 - \alpha^i)$. Also, in the precursor
state where $u$ was $u-1$, $i$ was already $i$. Hence the probability of being
in the precursor state was $Pr(i|u-1)$, and the probability of reaching the
final state via this route was $(1 - \alpha^i) \cdot Pr(i|u-1)$.

Now consider the case where $i$ {\em was} incremented, and was therefore $i-1$
when the $u$'th unique hash was processed. 
Because $i$ was incremented, the test in line 6 succeeded, so $h \geq \alpha^{i-1}$. 
Since $h$ is uniformly distributed on $[0,1)$, this happens with probability $\alpha^{i-1}$. 
Also, the precursor state was $(i-1,u-1)$, so the probability of reaching the
final state via this route was $\alpha^{i-1} \cdot Pr(i-1|u-1)$, and the 
total probability of reaching the final state $(i,u)$ is 
$(1 - \alpha^i) \cdot Pr(i|u-1) + \alpha^{i-1} \cdot Pr(i-1|u-1)$.






\noindent {\bf Technical Lemma with q=1:} The function
\[
g(1,k,u) = \sum_{i=0}^u \frac{1}{\alpha^i} \mathrm{Pr}(i|u)
\]
satisfies the following base case and recurrence:
\begin{align}
      g(1,k,0) = & 1 \\
g(1,k,u\!+\!1) = & g(1,k,u) + \left(\frac{1-\alpha}{\alpha}\right) \cdot 1
\end{align}
\begin{proof} The base case is true by inspection. The claimed recurrence is derived from equation ?? as follows:
\begin{align} 
  g(1,k,u\!+\!1)
= & \sum_{i=0}^{u+1} \frac{1}{\alpha^i} \mathrm{Pr}(i|u+1) \\
= & \left[ \sum_{i=0}^{u} \frac{1}{\alpha^i} (1-\alpha^i) \mathrm{Pr}(i|u) \right] + 0 \\
  & \quad + 0 + \left[ \sum_{i=1}^{u+1} \frac{1}{\alpha^i} (\alpha^{i-1}) \mathrm{Pr}(i\!-\!1|u) \right]
\end{align}
\noindent Now we will evaluate the two bracketed sums, repeatedly noting that probability distributions sum to 1.
For the first bracketed sum:
\begin{equation}
\sum_{i=0}^u \frac{1}{\alpha^i} \mathrm{Pr}(i|u) - \sum_{i=0}^u \frac{\alpha^i}{\alpha^i} \mathrm{Pr}(i|u) = g(1,k,u) - 1
\end{equation}

\noindent Next we evaluate the second bracketed sum, after performing the change of variables $j = i\!-\!1$:
\begin{equation}
\sum_{j=0}^{u} \frac{\alpha^j}{\alpha^{j+1}} \mathrm{Pr}(j|u) = \frac{1}{\alpha} \cdot 1
\end{equation}

Finally, combining ?? and ?? yields the claimed result:
\begin{equation}
  g(1,k,u\!+\!1) = g(1,k,u) - 1 + \frac{1}{\alpha} \cdot 1 = g(1,k,u) + \left(\frac{1-\alpha}{\alpha}\right) \cdot 1
\end{equation}
\end{proof}



